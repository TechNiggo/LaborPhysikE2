\section{Messprotokoll}
\label{sec:messprotokoll}
\pagestyle{messprotokoll}


%======================================%
%================Inhalt================%
\vspace{-0.5cm}
\begin{tabelle}
	\caption{Temperatur}
	\label{tab:messprotokoll->parameter}
	\begin{tabular}{|c|K{3.5\tablewidth}|K{3.5\tablewidth}|}
		\hline \rowcolor{firstcsvrow}
		Parameter & vor Versuchsdurchführung & nach Versuchsdurchführung \\ \hline
		Temperatur in $\grad C$ && \\\hline
		Luftdruck in $hPa$ && \\\hline
	\end{tabular}
\end{tabelle}


\subsection{Kennlinie und Wirkungsgrad der Elektrolysezelle}
\label{sub:messprotokoll->elektrolysezelle}

\vspace{-0.5cm}
\begin{tabelle}
	\caption{Faraday-Wirkungsgrad der Elektrolysezelle}
	\label{tab:messprotokoll->Elektrolyse->Wirkungsgrad}
	\begin{tabular}{|c|K{2.5\tablewidth}|K{2.5\tablewidth}|}
		\hline \rowcolor{firstcsvrow}
		Messung & $I$ in $A$ & $T(\Delta V=20~ml)$ in $s$ \\ \hline
		\leereZellen{3}{\csvcoli & &}
	\end{tabular}
\end{tabelle}

%Der Elektrolysestrom wurde auf \rule[-2pt]{1.2cm}{1pt} $A$ eingestellt. Innerhalb von \rule[-2pt]{1.2cm}{1pt} $s$ ist das Wasserstoffvolumen um $20~ml$ gestiegen.
 
\vspace{-0.5cm}
\begin{tabelle}
	\caption{Strom-Spannungs-Kennlinie der Elektrolysezelle}
	\label{tab:messprotokoll->Elektrolysezelle->Kennlinie}
	\begin{tabular}{|c|K{2.5\tablewidth}|K{2.5\tablewidth}|}
		\hline \rowcolor{firstcsvrow}
		Messwert & $I$ in $A$ & $U$ in $V$ \\ \hline
		\leereZellen{20}{\csvcoli & &}
	\end{tabular}
\end{tabelle}
Zusätzliche Bemerkungen:
\linie{4}


\subsection{Kennlinie und Wirkungsgrad der Brennstoffzellen}
\label{sub:messprotokoll->brennstoffzelle}
\vspace{-0.6cm}
\begin{tabelle}
	\caption{Nullmessung der Brennstoffzelle}
	\label{tab:messprotokoll->Brennstoffzelle->Nullmessung}
	\begin{tabular}{|c|K{2.5\tablewidth}|K{2.5\tablewidth}|}
		\hline \rowcolor{firstcsvrow}
		$\Delta V$ in $ml$ & $T$ in $s$ \\ \hline
		$2$ & \\ \hline
		$4$ & \\ \hline
		$6$ & \\ \hline
	\end{tabular}
\end{tabelle}

\vspace{-0.5cm}
\begin{tabelle}
	\caption{Strom-Spannungskennlinie der Brennstoffzelle}
	\label{tab:messprotokoll->Brennstoffzelle->Kennlinie}
	\begin{tabular}{|c|K{1.5\tablewidth}|K{1.5\tablewidth}|K{1.5\tablewidth}|K{1.5\tablewidth}|}
		\hline \rowcolor{firstcsvrow}
		Messwert & $I$ in $A$ & $R_{Last}$ in $\Omega$ & $U_{ges}$ in $V$ & $U_{uBZ}$ in $V$ \\ \hline
		\leereZellen{15}{\csvcoli &&&&}
	\end{tabular}
\end{tabelle}


\vspace{-0.5cm}
\begin{tabelle}
	\caption{Faraday-Wirkungsgrad (3-fache Durchführung)}
	\label{tab:messprotokoll->brennstoffzelle}
	\begin{tabular}{|c|K{0.88\tablewidth}|K{0.80\tablewidth}|K{0.80\tablewidth}||K{0.88\tablewidth}|K{0.80\tablewidth}|K{0.80\tablewidth}||K{0.88\tablewidth}|K{0.80\tablewidth}|K{0.80\tablewidth}|}
		\hline \rowcolor{firstcsvrow}
		$t$ in $s$ & \multicolumn{3}{|c|}{Messung 1} & \multicolumn{3}{|c|}{Messung 2} & \multicolumn{3}{|c|}{Messung 3} \\
		\hline \rowcolor{secondcsvrow}
		& $V$ in $ml$ & $I$ in $A$ & $U$ in $V$ & $V$ in $ml$ & $I$ in $A$ & $U$ in $V$ & $V$ in $ml$ & $I$ in $A$ & $U$ in $V$ \\ \hline
		
		\leereZellen{5}{& & & & & & & & &}
	\end{tabular}
\end{tabelle}
Zusätzliche Bemerkungen:
\linie{4}


\newpage
\kariert{47}

\newpage
\kariert{47}

\newpage
\kariert{47}