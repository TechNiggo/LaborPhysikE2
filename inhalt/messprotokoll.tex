\section{Messprotokoll}
\label{sec:messprotokoll}
\pagestyle{messprotokoll}


%======================================%
%================Inhalt================%
\begin{tabelle}
	\caption{Temperatur}
	\label{tab:messprotokoll->parameter}
	\begin{tabular}{|c|K{3.5\tablewidth}|K{3.5\tablewidth}|}
		\hline \rowcolor{firstcsvrow}
		Parameter & vor Versuchsdurchführung & nach Versuchsdurchführung \\ \hline
		Temperatur in $\grad C$ && \\\hline
		Luftdruck in $hPa$ && \\\hline
	\end{tabular}
\end{tabelle}


\subsection{Kennlinie und Wirkungsgrad der Elektrolysezelle}
\label{sub:messprotokoll->elektrolysezelle}

\begin{tabelle}
	\caption{Faraday-Wirkungsgrad der Elektrolysezelle}
	\label{tab:messprotokoll->Elektrolyse->Wirkungsgrad}
	\begin{tabular}{|c|K{2.5\tablewidth}|K{2.5\tablewidth}|}
		\hline \rowcolor{firstcsvrow}
		Messung & $I$ in $A$ & $T(\Delta V=20~ml)$ in $s$ \\ \hline
		\leereZellen{3}{\csvcoli & &}
	\end{tabular}
\end{tabelle}

%Der Elektrolysestrom wurde auf \rule[-2pt]{1.2cm}{1pt} $A$ eingestellt. Innerhalb von \rule[-2pt]{1.2cm}{1pt} $s$ ist das Wasserstoffvolumen um $20~ml$ gestiegen.

\begin{tabelle}
	\caption{Strom-Spannungs-Kennlinie der Elektrolysezelle}
	\label{tab:messprotokoll->Elektrolysezelle->Kennlinie}
	\begin{tabular}{|c|K{2.5\tablewidth}|K{2.5\tablewidth}|}
		\hline \rowcolor{firstcsvrow}
		Messwert & $I$ in $A$ & $U$ in $V$ \\ \hline
		\leereZellen{30}{\csvcoli & &}
	\end{tabular}
\end{tabelle}
Zusätzliche Bemerkungen:
\linie{2}


\subsection{Kennlinie und Wirkungsgrad der Brennstoffzellen}
\label{sub:messprotokoll->brennstoffzelle}
\begin{tabelle}
	\caption{Nullmessung der Brennstoffzelle}
	\label{tab:messprotokoll->Brennstoffzelle->Nullmessung}
	\begin{tabular}{|c|K{2.5\tablewidth}|K{2.5\tablewidth}|}
		\hline \rowcolor{firstcsvrow}
		$\Delta V$ in $ml$ & $T$ in $s$ \\ \hline
		$2$ & \\ \hline
		$4$ & \\ \hline
		$6$ & \\ \hline
	\end{tabular}
\end{tabelle}

\begin{tabelle}
	\caption{Strom-Spannungskennlinie der Brennstoffzelle}
	\label{tab:messprotokoll->Brennstoffzelle->Kennlinie}
	\begin{tabular}{|c|K{1.5\tablewidth}|K{1.5\tablewidth}|K{1.5\tablewidth}|K{1.5\tablewidth}|}
		\hline \rowcolor{firstcsvrow}
		Messwert & $I$ in $A$ & $R_{Last}$ in $\Omega$ & $U_{ges}$ in $V$ & $U_{uBZ}$ in $V$ \\ \hline
		\leereZellen{30}{\csvcoli &&&&}
	\end{tabular}
\end{tabelle}


\begin{tabelle}
	\caption{Strom-Spannungskennlinie der Brennstoffzelle (3-fache Durchführung)}
	\label{tab:messprotokoll->brennstoffzelle}
	\begin{tabular}{|K{\tablewidth}|K{\tablewidth}|l|K{\tablewidth}|K{\tablewidth}|r|K{\tablewidth}|K{\tablewidth}|}
		\cline {1-2}\cline {4-5}\cline {7-8}
		\multicolumn{2}{|c|}{\cellcolor{firstcsvrow} Messung 1} &~~~~~~& \multicolumn{2}{|c|}{\cellcolor{firstcsvrow} Messung 2} &~~~~~~& \multicolumn{2}{|c|}{\cellcolor{firstcsvrow} Messung 2} \\
		\cline {1-2}\cline {4-5}\cline {7-8}
		\cellcolor{secondcsvrow} $I$ in $A$ & \cellcolor{secondcsvrow} $U$ in $V$ &  &\cellcolor{secondcsvrow} $I$ in $A$ & \cellcolor{secondcsvrow} $U$ in $V$ &  & \cellcolor{secondcsvrow} $I$ in $A$ & \cellcolor{secondcsvrow} $U$ in $V$ \\
		\cline {1-2}\cline {4-5}\cline {7-8}
		\csvreader[separator=semicolon, late after line=\\\cline {1-2}\cline {4-5}\cline {7-8}, filter test=\ifnumless{\csvcoli}{44}]{./tables/hundert.csv}{}
			{& &  & & &  & &}
	\end{tabular}
\end{tabelle}
Zusätzliche Bemerkungen:
\linie{2}


\newpage
\kariert{47}

\newpage
\kariert{47}